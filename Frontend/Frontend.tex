\documentclass[11pt,a4paper]{book}
\usepackage[margin=2in]{geometry}
\usepackage[italian]{babel}
\usepackage[T1]{fontenc}
\usepackage[utf8]{inputenc}
\usepackage{graphicx}
\usepackage{imakeidx}
\usepackage{amsmath}
\usepackage{color}


\usepackage[hyperfootnotes=false, colorlinks=true, linkcolor=black]{hyperref}
\usepackage[style=numeric-comp,useprefix,hyperref,backend=bibtex]{biblatex}
\usepackage{listings}  % Serve per evidenziare i blocchi di codice
\usepackage{pxfonts} % permette di avere caratteri in lstlisting con formattazione
\usepackage[normalem]{ulem}
\useunder{\uline}{\ul}{}

\setlength{\parskip}{1em} % cambia l'interlinea prima di un nuovo capoverso

% Colori per lstlisting
\definecolor{pblue}{rgb}{0.13,0.13,1}
\definecolor{pgreen}{rgb}{0,0.5,0}
\definecolor{pred}{rgb}{0.9,0,0}
\definecolor{pgrey}{rgb}{0.46,0.45,0.48}
\definecolor{maroon}{rgb}{0.5,0,0}
\definecolor{plightgrey}{rgb}{0.8,0.8,0.8} 
\definecolor{darkblue}{rgb}{0.0,0.0,0.6}
\definecolor{cyan}{rgb}{0.0,0.6,0.6}

\usepackage{xcolor} % Necessario per definire i colori
\hypersetup{
  colorlinks=true,
  linkcolor=green!70!black,
  urlcolor=green!70!black
} % Setup colore link


\lstset{ % Riduce la larghezza della tabulazione per lstlisting
  tabsize=2,
  backgroundcolor=\color{plightgrey},
  breaklines=true,
  postbreak=\mbox{\textcolor{red}{$\hookrightarrow$}\space},
  columns=fullflexible,
  frame=single,
}


\lstset{
  language=XML,
  basicstyle=\ttfamily,
  morestring=[s]{"}{"},
  morecomment=[s]{?}{?},
  morecomment=[s]{!--}{--},
  commentstyle=\color{white},
  moredelim=[s][\color{black}]{>}{<},
  moredelim=[s][\color{red}]{\ }{=},
  stringstyle=\color{blue},
  identifierstyle=\color{maroon}
}
% Crea un set di impostazioni per Java in lstlisting
\lstset{language=Java,
  showspaces=false,
  showtabs=false,
  breaklines=true,
  showstringspaces=false,
  breakatwhitespace=true,
  commentstyle=\color{white},
  keywordstyle=\color{pblue},
  stringstyle=\color{pred},
  basicstyle=\ttfamily,
  moredelim=[is][\textcolor{pgrey}]{\%\%}{\%\%}
}

% Crea un set di impostazioni per C++ in lstlisting
\lstset{language=C++,
                basicstyle=\ttfamily,
                keywordstyle=\color{blue}\ttfamily,
                stringstyle=\color{red}\ttfamily,
                commentstyle=\color{white}\ttfamily,
                morecomment=[l][\color{magenta}]{\#}
}

\lstdefinelanguage{XML}{
  morestring=[b]",
  morestring=[s]{>}{<},
  morecomment=[s]{<?}{?>},
  stringstyle=\color{black},
  identifierstyle=\color{darkblue},
  keywordstyle=\color{cyan},
  morekeywords={xmlns,version,type}% list your attributes here
}

\begin{document}
\title{Frontend}
\author{Jacopo De Angelis}
\maketitle

\pagebreak
\tableofcontents
\pagebreak

\chapter{HTML5 e CSS3}
\section{HTML5}
\subsection{Tag}
\paragraph{<!--...-->}
\begin{lstlisting}[language = HTML]
<!--This is a comment. Comments are not displayed in the browser-->
\end{lstlisting}
Serve per i commenti nei file HTML e CSS

\paragraph{<!--... //-->}
\begin{lstlisting}[language = HTML]
<!--
function displayMsg() {
  alert("Hello World!")
}
//-->
\end{lstlisting}
Serve per i commenti nei file JS

\paragraph{<!DOCTYPE>}
\begin{lstlisting}[language = HTML]
<!DOCTYPE html>
<html>
<head>
<title>Title of the document</title>
</head>

<body>
The content of the document......
</body>

</html>
\end{lstlisting}
Deve essere il primo tag di ogni documento TML, serve per segnalare al browser che si troverà dover interpretare una pagina HTML. Questa dichiarazione è unica in HTML5.

\paragraph{<a>}
\begin{lstlisting}[language = HTML]
<a href="https://www.w3schools.com">Visit W3Schools.com!</a>
\end{lstlisting}
Definisce un link.
% Please add the following required packages to your document preamble:
% \usepackage[normalem]{ulem}
% \useunder{\uline}{\ul}{}
\begin{table}[]
\begin{tabular}{|l|l|l|}
\hline
{\ul \textbf{Attributo}} & {\ul \textbf{Valore}}                                                                                                                                             & {\ul \textbf{Descrizione}}                                                                                                                                                                 \\ \hline
\textbf{download}        & Nome del file                                                                                                                                                     & Specifica che il link verrà scaricato una volta premuto                                                                                                                                    \\ \hline
\textbf{href}            & URL                                                                                                                                                               & Specifica l'URL di una pagina                                                                                                                                                              \\ \hline
\textbf{hreflang}        & codice ISO                                                                                                                                                        & Specifica il linguaggio del documento                                                                                                                                                      \\ \hline
\textbf{media}           & media\_query                                                                                                                                                      & \begin{tabular}[c]{@{}l@{}}Specifica  per quale tipo di device il documento \\ collegato è ottimizzato\end{tabular}                                                                        \\ \hline
\textbf{ping}            & lista di URL                                                                                                                                                      & \begin{tabular}[c]{@{}l@{}}Specifica una lista di URL separate da spazi alla \\ quale il link è collegato.Tutti quei link verranno pingati, \\ solitamente usato per tracking\end{tabular} \\ \hline
\textbf{referrerpolicy}  & \begin{tabular}[c]{@{}l@{}}no-referrer\\ no-referrer-when-downgrade\\ origin\\ origin-when-cross-origin\\ unsafe-url\end{tabular}                                 & Specifica quale riferimento mandare                                                                                                                                                        \\ \hline
\textbf{rel}             & \begin{tabular}[c]{@{}l@{}}alternate\\ author\\ bookmark\\ external\\ help\\ license\\ next\\ nofollow\\ noreferrer\\ noopener\\ prev\\ search\\ tag\end{tabular} & \begin{tabular}[c]{@{}l@{}}Specifica la relazione tra il documento corrente\\ e quello collegato\end{tabular}                                                                              \\ \hline
\textbf{target}          & \begin{tabular}[c]{@{}l@{}}\_blank\\ \_parent\\ \_self\\ \_top\\ framename\end{tabular}                                                                           & Specifica dove aprire il documento                                                                                                                                                         \\ \hline
\end{tabular}
\end{table}

\paragraph{<abbr>}
\begin{lstlisting}[language = HTML]
The <abbr title="World Health Organization">WHO</abbr> was founded in 1948.
\end{lstlisting}
Definisce un acronimo.

\paragraph{<address>}
\begin{lstlisting}[language = HTML]
<address>
	Written by <a href="mailto:webmaster@example.com">Jon Doe</a>.<br>
	Visit us at:<br>
	Example.com<br>
	Box 564, Disneyland<br>
	USA
</address>
\end{lstlisting}
Definisce definisce le informazioni dell'autore di un documento. A seconda di dove è inserito definisce l'autore di parti differenti:
\begin{itemize}
	\item body: l'autore del documento
	\item article: autore dell'articolo
\end{itemize}


\paragraph{<area>}
\begin{lstlisting}[language = HTML]
<img src="planets.gif" width="145" height="126" alt="Planets"
usemap="#planetmap">

<map name="planetmap">
  <area shape="rect" coords="0,0,82,126" href="sun.htm" alt="Sun">
  <area shape="circle" coords="90,58,3" href="mercur.htm" alt="Mercury">
  <area shape="circle" coords="124,58,8" href="venus.htm" alt="Venus">
</map>
\end{lstlisting}

% Please add the following required packages to your document preamble:
% \usepackage[normalem]{ulem}
% \useunder{\uline}{\ul}{}
\begin{table}[]
\begin{tabular}{|l|l|l|}
\hline
{\ul \textbf{Attributo}} & {\ul \textbf{Valore}}                                                                                                                                             & {\ul \textbf{Descrizione}}                                                                                          \\ \hline
\textbf{alt}             & Testo                                                                                                                                                             & Specifica un testo alternativo                                                                                      \\ \hline
\textbf{coords}          & coordinate                                                                                                                                                        & Specifica le coordinate dell'area                                                                                   \\ \hline
\textbf{download}        & Nome del file                                                                                                                                                     & Specifica che il link verrà scaricato una volta premuto                                                             \\ \hline
\textbf{href}            & URL                                                                                                                                                               & Specifica l'URL di una pagina                                                                                       \\ \hline
\textbf{hreflang}        & codice ISO                                                                                                                                                        & Specifica il linguaggio del documento                                                                               \\ \hline
\textbf{media}           & media\_query                                                                                                                                                      & \begin{tabular}[c]{@{}l@{}}Specifica  per quale tipo di device il documento \\ collegato è ottimizzato\end{tabular} \\ \hline
\textbf{rel}             & \begin{tabular}[c]{@{}l@{}}alternate\\ author\\ bookmark\\ external\\ help\\ license\\ next\\ nofollow\\ noreferrer\\ noopener\\ prev\\ search\\ tag\end{tabular} & \begin{tabular}[c]{@{}l@{}}Specifica la relazione tra il documento corrente\\ e quello collegato\end{tabular}       \\ \hline
\textbf{shape}           & \begin{tabular}[c]{@{}l@{}}default\\ rect\\ circle\\ poly\end{tabular}                                                                                            & Specifica la forma dell'area                                                                                        \\ \hline
\textbf{target}          & \begin{tabular}[c]{@{}l@{}}\_blank\\ \_parent\\ \_self\\ \_top\\ framename\end{tabular}                                                                           & Specifica dove aprire il documento                                                                                  \\ \hline
\textbf{type}            & Tipo di media                                                                                                                                                     & Specifica il tipo di media dell'URL                                                                                 \\ \hline
\end{tabular}
\end{table}

\paragraph{<article>}
Specifica un contenuto indipendente, auto contenuto. Dovrebbe aver senso di per sè e dovrebbe essere distribuibile 

\paragraph{<aside>}
Indica un contenuto a parte rispetto al contenuto nel quale è inserito

\paragraph{<audio>}
Contiene audio stream. Al momento sono supportati tre formati: MP3, WAV e OGG.

% Please add the following required packages to your document preamble:
% \usepackage[normalem]{ulem}
% \useunder{\uline}{\ul}{}
\begin{table}[]
\begin{tabular}{|l|l|l|}
\hline
{\ul \textbf{Attributo}} & {\ul \textbf{Valore}}                                          & {\ul \textbf{Descrizione}}                                                                                                                 \\ \hline
\textbf{autoplay}        & autoplay                                                       & Specifica che l'audio parte appena pronto                                                                                                  \\ \hline
\textbf{controls}        & controls                                                       & Specifica che i controlli devono essere mostrati                                                                                           \\ \hline
\textbf{loop}            & loop                                                           & Specifica che l'audio partirà nuovamente                                                                                                   \\ \hline
\textbf{muted}           & muted                                                          & Specifica che l'output sarà mutato                                                                                                         \\ \hline
\textbf{preload}         & \begin{tabular}[c]{@{}l@{}}auto\\ metadata\\ none\end{tabular} & \begin{tabular}[c]{@{}l@{}}Specifica se e come l'autore pensa che l'audio\\ debba essere caricato quando la pagina è caricata\end{tabular} \\ \hline
\textbf{src}             & URL                                                            & Specifica l'URL della fonte                                                                                                                \\ \hline
\end{tabular}
\end{table}

\paragraph{<base>}
Inserito nell'head, specifica l'URL base per tutti i link della pagina.

\paragraph{<bdi>}
Bi-Direction Isolation, serve per separare dal resto del testo una parte che potrebe essere scritta in una direzione differente rispetto al resto.	

\paragraph{<bdo>}
dir = rtl/ltr
Bi-Direction Override, serve per separare dal resto del testo una parte che potrebe essere scritta in una direzione differente rispetto al resto e imporre la direzione.	

\paragraph{<blockquote>}
Serve per citare un'altra fonte. cite=url per dare l'origine.
<q> per brevi citazioni inline.

\subsection{Style}
Style, tag utilizzato all'interno di head, serve ad includere l'editing dello stile della pagine. Lo si può fare all'interno della pagina stessa ma è una pratica altamente sconsigliata, sarebbe meglio creare un file .css esterno da poter includere.

La pratica migliore è usare link con un riferimento al file css esterno.

\begin{center}
	\begin{huge}
		Codice HTML
	\end{huge}
\end{center}
\lstinputlisting[language=HTML]{Codici\HTML\Lezione_3.html}

\begin{center}
	\begin{huge}
		Codice CSS
	\end{huge}
\end{center}
\lstinputlisting[language=HTML]{Codici\CSS\Lezione_3.css}
\chapter{CSS}
\chapter{Javascript}
\section{Variabili}
Si crea una variabile con "var". JS è debolmente tipizzato.

I modi migliori per definire una variabile sono due:
\begin{itemize}
	\item \textbf{let}: 
	\item \textbf{const}: 
\end{itemize}
\chapter{Link utili}
\begin{itemize}
	\item \href{https://www.w3schools.com/}{W3 school}
	\item \href{https://developer.mozilla.org/it/}{Mozilla developer guide}
\end{itemize}

\end{document}


