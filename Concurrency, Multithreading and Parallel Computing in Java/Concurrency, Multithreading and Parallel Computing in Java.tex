\documentclass[11pt,a4paper]{book}

\usepackage{Appunti}

\begin{document}
\title{Concurrency, Multithreading and Parallel Computing in Java}
\author{Jacopo De Angelis}
\maketitle

\pagebreak
\tableofcontents
\pagebreak

\chapter{Multhithreading theory}
\section{Processi e thread}
I processi e i thread sono indipendenti tra di loro:
\begin{itemize}
	\item i \textbf{processi} sono l'istanza di esecuzione: ogni processo ha registri indipendenti, uno stack di memoria e un heap per ogni procesos. In java si crea con \textbf{ProcessBuilder}.
	\item i \textbf{thread} sono l'unità di esecuzione di un processo: un processo può avere più thread, ogni thread condivide memoria e risorse.
\end{itemize}

\section{Time slicing}
Il tempo di funzionamento di un core è diviso tra i suoi thread, il modo nel quale il tempo è diviso tra di essi è deciso tramite un algoritmo di time slicing.

\section{Vantaggi del multithreading}
La possibilità di svolgere più lavori contemporaneamente senza dovere attendere la fine di uno per iniziare l'altro.

\section{Svantaggi del multithreading}
Problemi di sincronizzazione e di utilizzo delle risorse possono rendere il multithreading inefficiente. Il debug è più complicato.

\section{Ciclo di vita di un thread}
\begin{itemize}
	\item instanziazione (new): quando viene creato
	\item in funzione (runnable): quando viene eseguito tramite start()
	\item in attesa (waiting): passaggio ad uno stato di attesa tramite wait() e sleep()
	\item fine vita (dead): quando il thread termina la sua funzione
\end{itemize}

\chapter{Manipolazione dei thread}
\section{Interfaccia Runnable}
\href{https://docs.oracle.com/javase/7/docs/api/java/lang/Runnable.html}{Runnable} porta ad implementare una funzione run() dentro alla quale viene posta la parte logica del thread. Il thread a questo punto viene eseguito tramite start()

\section{Classe Thread}
Estendendo la classe \href{https://docs.oracle.com/javase/7/docs/api/java/lang/Thread.html}{Thread} si ottiene lo stesos risultato ma con il limite di non potere estendere altre classi, si ottengono però i suoi metodi.

\section{join()}
\emph{thread.join()} attende la fine dell'esecuzione del thread. Serve per segnalare quando il thread muore.

\section{Deamon e Worker}
\begin{itemize}
	\item \textbf{Deamon}: è un thread che vive durante tutta la vita del \textbf{processo}. Sono thread di aiuto, ad esempio il garbage collector. 
	\item \textbf{Worker}: è un thread che vive durante la propria vita
\end{itemize}

\chapter{Comunicazione tra thread}
Le variabili locali vivono sullo stack, gli oggetti che sono instanziati vivono sull'heap. \textbf{Ogni thread ha il suo stack ma tutti i thread condividono l'heap.} 

\end{document}