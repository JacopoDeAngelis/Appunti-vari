\documentclass[11pt,a4paper]{book}

\usepackage{Appunti}

\begin{document}
\title{Secure code}
\author{Jacopo De Angelis}
\maketitle

\pagebreak
\tableofcontents
\pagebreak

\chapter{Concetti di sicurezza}
\section{Privilegi minimi}\label{sec: privilegi minimi}
Nel caso i permessi non siano dati correttamente, un utente malintenzionato potrebbe decidere di accedere ai dati in maniera distruttiva, ad esempio tramite una SQL injection.

In questo caso una soluzione è cambiare i permessi per i vari gruppi in modo che abbiano il potere minimo per le loro funzioni. Ad esempio, nel caso di un database di un sito dove le query non sono sanificate (ovvero è possibile eseguire codice di accesso al DB all'interno delle query), si può decidere che gli utenti abbiano solo possibilità di lettura e non di scrittura.

Nel caso l'utente malintenzionato acceda tramite un ruolo da amministratore potrebbe decidere di cambiare password dei database, cambiare dati o interrompere il servizio. In più qualsiasi codice eseguito da un utente con certi privilegi verrebbe eseguito con gli stessi, causando altri tipi di malfunzionamento

\textbf{Il concetto dei privilegi minimi non corregge le vulnerabilità del sistema, come le code injection, ma rende più difficile per un utente malintenzionato sfruttare queste falle}

\textbf{In generale i processi avviati dall'applicazione dovrebbero essere eseguiti con i privilegi minimi necessari}

\section{Difesa in profondità}\label{sec: difesa in profondità}
Il concetto dietro alla difesa in profondità è quello di creare più livelli di sicurezza, di diverso tipo, tramite diversi sistemi. Ad esempio:
\begin{itemize}
	\item Dividere i server in più network in modo che l'attacco non possa propagarsi
	\item inserire un firewall
	\item criptare i dati del database
	\item sanificare le query e parametrizzarle
	\item usare validazione sia a livello di frontend che di backend
	\item mettere in sicurezza i dati nel mondo fisico, ad esempio tramite guardie ai server, accesso solo tramite badge, ecc.
\end{itemize}

\textbf{Implementare sempre più livelli di sicurezza, di più tipi, fisici, architetturali e di codice}

\textbf{Livello dati}: controllo degli accessi, criptazione, backup

\textbf{Livello applicazione}: autenticazione, autorizzazione, auditing (noti assieme come AAA), codice sicuro e "hardening"

\textbf{Livello host}: "hardening", autenticazione, gestione degli aggiornamenti, antivirus

\textbf{Livello di network interno}: Segmentazione del network, IPsec, TLS, NAT

\textbf{Livello perimetrale}: firewall, TLS, negazione dell'accesso, prevenzione

\textbf{Sicurezza fisica}: guardie, lucchetti, dispositivi di tracciamento, badge
\section{Sicuro per default}\label{sec: sicuro per default}
Il concetto dietro ad un software sicuro per default è che le impostazioni di sicurezza siano correttamente impostate.

Per fare un esempio, la richiesta affinchè la password inserita sia sicura (lunghezza minima, caratteri speciali, ecc.) è una buona misura di sicurezza che dovrebbe essere implementata di base. In questo modo gli utenti sarebbero già più al sicuro da attacchi di brute force.

Un altro esempio è quella di una code injection in campi non classici, ad esempio tramite l'inserimento password. Per questo caso si potrebbe implementare una sanificazione delle query al sistema, parametrizzando le richieste e impedendo che queste abbiano effetti indesiderati. In più, basandoci sul concetto dei privilegi minimi (\ref{sec: privilegi minimi}) le query potrebbero essere eseguite con i privilegi minimi.

\textbf{Uno sviluppatore dovrebbe pensare alla sicurezza dall'inizio e durante tutto lo sviluppo}

\textbf{Applicare il concetto della difesa in profondità (\ref{sec: difesa in profondità}), ovvero implementare più livelli di sicurezza}

\textbf{Implementare la sicurezza a livello di applicazione e di infrastruttura}

\textbf{Disabilitare funzionalità o servizi non usati}

\section{Controllo di errori robusto}
Mai dare troppe informazioni all'utente riguardanti un errore. Ad esempio, mai permettere che una pagina, come errore, restituisca uno stack dell'errore o altre informazioni che potrebbe usare.

Parte essenziale di questo controllo è la gestione delle eccezioni. Ad esempio se una pagina di log in lanciasse un'eccezione causata da un input non previsto potrebbe semplicemente restituire un messaggio generico di errore. 

\textbf{Usare messaggi di errore generici e controllare la chiusura dell'applicazione}

\textbf{Non condividere informaizoni private dell'applicazione}

\textbf{Scrivere su di un log gli errori in modo da poterli usare per le correzioni}

\section{Non fidarsi di alcun input \textit{(trust no input)}}
Le applicazioni non dovrebbero mai accettare un input non validato dal sistema. Ad esempio un errore che deriva da un precedente errore e che fa crollare l'app.

Ricevere --> validare --> calcolare --> gestire le eccezioni

\textbf{Mai fidarsi dell'utente}

\textbf{Limitare le opzioni possibili per l'utente, ad esempio offrendo un menù a tendina per la selezione}

\textbf{Validare ogni input, inclusi gli input che arrivano da file e DB}

\textbf{Rifiutare i dati invalidi o almeno depurarli o ignorarli}

\textbf{Esempi di validazione}:
\begin{itemize}
	\item corrispondenza
	\item whitelist
	\item blacklist
\end{itemize}


\section{Open design}
Quando ci si basa sul design dell'applicazione come misura di sicurezza si è in errore. Si dice anche "sicurezza per oscuramento". Ad esempio, usare una porta non comune non è una buona misura di sicurezza, si possono scansionare le porte.

Anche un'autenticazione nascosta, ad esempio passando un form segreto per indicare il livello di privilegio dell'utente, potrebbe essere sfruttato.

Si potrebbero usare certificati, oppure usare ruoli legati all'utente.

\textbf{Come regola base si dovrebbe implementare tutta la sicurezza}

\textbf{Non basarsi solo sulla sicurezza per oscuramento}

\textbf{La sicurezza si dovrebbe basare su metodi conosciuti}


\section{Fallire in sicurezza}
Le eccezioni dovrebbero essere tutte gestite in sicurezza.

\textbf{Ogni blocco dovrebbe avere tre risultati determinati}:
\begin{itemize}
	\item SE l'utente è autorizzato ALLORA esegui l'azione
	\item SE l'utente non è autorizzato ALLORA non eseguire l'azione
	\item SE c'è un'eccezione ALLORA torna a prima dell'azione e manda un messaggio di errore
\end{itemize}

\section{Riutilizzo dei sistemi di sicurezza}
Invece di creare nuovo codice (e quindi nuove falle) sarebbe meglio riutilizzare, dove possibile, codice già in uso e già testato.

Se, ad esempio, un modulo viene utilizzato in più punti e poi viene scoperta una falla, basterà correggere il modulo per applicare la modifica a tutti i blocchi di codice. Avessimo creato del codice duplicato ci sarebbero stati problemi.

\textbf{Come regola, tenere la semplicità in mente quando si crea un sistema}

\textbf{Sempre meglio riutilizare codice già esistente e testato}

\textbf{Educare il gruppo alle buone pratiche di programmazione}

\section{Logging}
Mai loggare informazioni confidenziali o personali. Sbagliato anche NON loggare mai. É sbagliato anche non loggare informazioni utili dal punto di vista della sicurezza.

I log possono essere usati per permettere la risposta ad un eventuale attacco di brute forcing, ad esempio.

\textbf{Mai centralizare i log. Assicurarsi che i log siano semplici da vedere e gestire}

\textbf{Bisognerebbe loggare a tutti i livelli. Eventi come tentativi di login, modifiche o richieste di dati dovrebbero essere loggate}

\textbf{Seguire le 5 W del logging}:
\begin{enumerate}
	\item Cos'è successo
	\item Chi era coinvolto
	\item Quando
	\item Dove, inteso come informazioni sul luogo fisico di attacco
	\item Da dove arriva il problema
\end{enumerate}

\textbf{Non loggare informazioni private}

\textbf{Limitare l'accesso ai log in base all'utente}

\section{Protezione dei dati}
Il codice e le informazioni importanti andrebbero protette in più modi, ad esempio:
\begin{itemize}
	\item comunicazioni protette tramite tls
	\item niente cache dei dati privati, come ad esempio alle carte di credito
	\item la repo è su di un server criptato al sicuro
	\item accesso solo interno
	\item accesso limitato agli utenti necessari
	\item codice offuscato
\end{itemize}

Anche proteggere la privacy è essenziale. I dati possono essere protetti tramite accesso limitato ai dati, contratti di non diffusione, ecc.

\chapter{Sicurezza web app 101}
\section{HTTP header insufficienti}
Questa categoria è quando un header è impostato in maniera poco sicura perchè mancano alcune chiavi per mitigare certi tipi di attacco o direttamente certe voci.

Ad esempio, se mancasse l'opzione X-frame si ptorebbe creare un click jacking, ovvero, poter reindirizzare i click tramite layer nascosti. Sarebbe peggio se non venissero validati degli script inseriti tramite phishing.

Oppure se si utilizzano vecchie versioni di certi strumenti, lasciando aperte certe falle. COse come accedere agli stream come man in the middle o cross-site scripting sono possibili proprio per queste mancanze.

Si possono prevenire certi problemi stando attenti ad esempio:
\begin{itemize}
	\item che si utilizzino le ultime implementazioni sicure
	\item prestare attenzione ad alcuni header, tipo:
	\begin{itemize}
		\item X-XSS-Protection: abilità i filtri XSS
		\item X-Content-Tyoe-Options: previene che si possa effettuare \href{https://www.keycdn.com/support/what-is-mime-sniffing}{MIME sniffing} alla reponse di un altro contenuto
		\item Content-Security-Policy: copre svariate falle
		\item X-Permitted-Cross-Domain-Polices: limita l'accesso ai dati da parte di Flash Player
		\item Content-Type: limita i formati che possono interagire con la risorsa
		\item X-Frame-Options: disabilità il caricamento di una risorsa in un iframe fuori dal suo dominio
		\item Rimuovere tutti i X-Powered-By headers
	\end{itemize}
\end{itemize}


\chapter{\href{https://owasp.org/www-community/attacks/}{OWASP}}
\section{Binary planting}
è un nome generale per un attacco nel quale un hacker piazza un file contenente del codice maligno in un sistema remoto o locale in modo da sfruttare una vulnerabilità e caricarlo. Questo attacco può accadere in vari modi:
\begin{itemize}
	\item Permessi d'accesso insicuri ad una cartella locale per permettere ad un attaccante di inserire del codice maligno in punti autorizzati
	\item Un'applicazione potrebbe avere accesso a zone per inserire il codice maligno\footnote{Chissà perchè IE è visto così male, vero?}
%	\item L'applicazione cerca ilcodice binario in posizioni non sicure, ad esempio in un sistema remoto, ad esempio un'applicazione windows che carica un DLL\footnote{Dynamic Link Library} dalla cartella corrente dopo che è stata resa una cartella condivisa
\end{itemize}
\section{Injection flaws}
\subsection{SQL injection cieca}
É un un tipo di attacco nel quale vengono fatte domande al server di tipo vero o falso e si determina la risposta in base alla risposta del server. Questo attacco viene usato spesso per le applicazioni web che mostrano errori generici ma non sono mitifate per le vulnerabilità SQL.

Viene usato quando il server non mostra dati con una richiesta classica e allora bisogna trovare un modo per arginare il problema.
\subsection{SQL injection}
\subsection{XPath injection cieca}
Xpath è un tipo di linguaggio di query, quindi è simile a SQL come idea. La cosa che cambia è che vada a cercare all'interno di un qualsiasi file XML senza restrizioni in quanto la nozione di Utente non esiste. L'attacco è quindi simile alla SQL injection cieca, semplicemente non ci sono restrizioni in corso.
\section{Brute force}
Questo attacco si può manifestare in molti modi ma primariamente consiste nel configurare dei valori prestabiliti e continuare a fare richieste al server con quei valori per poi analizzare le risposte. Si potrebbe usare un attacco basato su di un dizionario.
\section{Buffer overflow tramite variabili d'ambiente}
Una volta che l'hacker comprendere come modificare le variabili d'ambiente, potrebbe provare a creare un overflow sul buffer collegato. Per questo attacco ci sono dei requisiti:
\begin{itemize}
	\item l'applicazione deve usare variabili d'ambiente
	\item le variabili d'ambiente sono esposte all'utente e sono vulnerabili per un buffer overflow
	\item l'ambiente vulnerabile usa dati non sicuri
	\item i dati non sono propriamente validati
\end{itemize}

\section{Buffer overflow}
COnsiste nel modificare il buffer dei dati riscrivendo certi segmenti di memoria o dei processi, generando errori a catena.

\section{CORS\footnote{Cross-Origin Resource Sharing} OriginHeaderScrutiny}
É una feature che permette di:
\begin{itemize}
	\item Per un'applicazione web di esporre risorse a tutti o ad un dominio ristretto
	\item Ad un web client di eseguire richieste AJAX per risorse di altri domini
\end{itemize}

Il rischio qua è che venga inserito un qualsiasi valore all'interno dell'header HTTP Origin per forzare un'applicazione web ad offrire il contenuto richiesto. 

\section{CORS RequestPreflighScrutiny}


\section{Password enumeration}
É un attacco nel quale l'hacker prova a comprendere se un utente è all'interno del sistema o no.

\section{Session handling}
\section{Autenticazione}
\section{Business logic}

\section{•}
\end{document}