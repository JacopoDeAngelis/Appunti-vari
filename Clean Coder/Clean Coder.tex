\documentclass[11pt,a4paper]{book}

\usepackage{Appunti}

\begin{document}
\title{Clean Coder\\
\large{\textit{Robert C. Martin}}}
\author{Jacopo De Angelis}
\maketitle

\pagebreak
\tableofcontents
\pagebreak

\chapter{Professionalità}
\section{Prendere le proprie responsabilità}
Come si può "imparare" a prendersi le proprie responsabilità? Ci sono alcuni principi da seguire.

\section{Non fare del male}
Come può un programmatore fare del male? Può farlo al software, creando problemi per le funzioni e alla struttura del software.
\subsection{Non danneggiare la funzionalità}
Chiaramente il software deve funzionare, non solo per noi programmatori ma anche per clienti e datori di lavoro. Per essere professionali, insomma, non bisogna creare bug.
\begin{figure}[h!]
	\begin{center}
		\includegraphics[scale=0.3]{img/001.jpg}
		\caption{Tu in questo momento}
		\label{fig: 001}
	\end{center}
\end{figure}
Il concetto è che quello deve essere l'impegno di un programmatore e nel caso escano dei bug deve prendersene la responsabilità se causati dal proprio codice.

\subsection{I QA non dovrebbero trovare niente}
Qual è il codice per il quale non sai se i QA\footnote{Quality assurance, coloro incaricati di testare il codice} troveranno qualcosa? Il codice di cui non si è certi.

Usare i QA come dei cacciatori di bug rende il processo più lungo e riduce la fiducia nei programmatori. Mandare in testing codice di cui non si è sicuri è violare la regola del "non fare del male".

I QA troveranno errori nel codice? Probabile.

\subsection{Devi sapere che funziona}
Come? Semplice, testando il codice.\\
Paura di metterci troppo? Automatizzali scrivendo degli unit test.\\
Quanto codice andrebbe testato? Tutto.\\

\subsection{QA automatizzato}


\end{document}